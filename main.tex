%%%%%%%%%%%%%%%%%%%%%%%%%%%%%%%%%%%%%%%%%
% Focus Beamer Presentation
% LaTeX Template
% Version 1.0 (8/8/18)
%
% This template has been downloaded from:
% http://www.LaTeXTemplates.com
%
% Original author:
% Pasquale Africa (https://github.com/elauksap/focus-beamertheme) with modifications by 
% Vel (vel@LaTeXTemplates.com)
%
% Template license:
% GNU GPL v3.0 License
%
% Important note:
% The bibliography/references need to be compiled with bibtex.
%
%%%%%%%%%%%%%%%%%%%%%%%%%%%%%%%%%%%%%%%%%

%----------------------------------------------------------------------------------------
%	PACKAGES AND OTHER DOCUMENT CONFIGURATIONS
%----------------------------------------------------------------------------------------

\documentclass{beamer}

\usetheme{focus} % Use the Focus theme supplied with the template
% Add option [numbering=none] to disable the footer progress bar
% Add option [numbering=fullbar] to show the footer progress bar as always full with a slide count

% Uncomment to enable the ice-blue theme
%\definecolor{main}{RGB}{92, 138, 168}
%\definecolor{background}{RGB}{240, 247, 255}

%------------------------------------------------

\usepackage{booktabs}% Required for better table rules
\usepackage[style=authoryear, backend=biber, hyperref,sorting=none]{biblatex}
\addbibresource{references.bib}

%----------------------------------------------------------------------------------------
%	 TITLE SLIDE
%----------------------------------------------------------------------------------------

\title{Introduction to Version Control \& Git}

\subtitle{Teach Me X}

\author{Ewan Gilligan}

\titlegraphic{\includegraphics[scale=0.15]{Images/Git-Logo-Black.png}}



\date{\today}

%------------------------------------------------

\begin{document}


%------------------------------------------------

\begin{frame}
	\maketitle % Automatically created using the information in the commands above
\end{frame}

\begin{frame}{Motivation}
What is Version Control?
    \begin{itemize}
        \item ``Version control is a system that records changes to a file or set of files over time so that you can recall specific versions later'' \cite{10.5555/2695634}
        % A very naive way of doing this is to have myfile1.txt, myfile2.txt, myfileFINAL.txt. Very cumbersome, many files, and prone to error.
    \end{itemize}
And why is it important?
    \begin{itemize}
        \item Revert changes that mess up your codebase
        \item Humans are bad at managing versions
        \item Separate changes that different users make
        \item Find out who made which changes
        \item ...
    \end{itemize}
\end{frame}

\begin{frame}{Git}
What it is?
\begin{itemize}
    \item Distributed Version Control System (More on this in a second)
    % Previously BitKeeper was used, but they were a commercial company and free of charge use was revoked.
    \item Developed for use with the Linux Kernel in 2005
\end{itemize}
Why you should use it?
\begin{itemize}
    \item Most commonly used Version Control System (VCS) in the world (according to the Stack Overflow Dev Survey)
    \item Commonly used in both open source and commerical software development.
\end{itemize}
    
\end{frame}

\begin{frame}{Aside: Local, Distributed, and Centralised VCS}
    Local
    \begin{itemize}
        \item Keeping track of changes to files locally only.
    \end{itemize}
    % Simple, no extra setup required. 
    Centralised
    \begin{itemize}
        \item Single server that contains all versions files
        \item Clients check out files from the central server
    \end{itemize}
    % Much easier to administrate
    % Single point of failure, same issue for local
    Distributed
    \begin{itemize}
        \item Client mirrors entire repository, which includes the history.
    \end{itemize}
    % Each clone of the system is essentially a backup.
    % Git can be used entirely locally, or as distributed.
\end{frame}

\begin{frame}{Basic Git Workflow}
    
\end{frame}

\begin{frame}{How to ``git'' help}
There are a two main ways to get help with git.
\begin{itemize}
    \item To get detailed help check the \texttt{man} page
    %either man git <verb>, git help <verb>,  git <verb> -h
    \item For a quick overview add the \texttt{-h} option to the command, e.g. \texttt{git add -h}
    \item Stack Overflow
    % Almost any issue you have with git is likely to have been encountered before.
    
\end{itemize}
\end{frame}

\begin{frame}{References}
    \printbibliography[heading=none]
\end{frame}



\end{document}
