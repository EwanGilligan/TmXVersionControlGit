%%%%%%%%%%%%%%%%%%%%%%%%%%%%%%%%%%%%%%%%%
% Focus Beamer Presentation
% LaTeX Template
% Version 1.0 (8/8/18)
%
% This template has been downloaded from:
% http://www.LaTeXTemplates.com
%
% Original author:
% Pasquale Africa (https://github.com/elauksap/focus-beamertheme) with modifications by 
% Vel (vel@LaTeXTemplates.com)
%
% Template license:
% GNU GPL v3.0 License
%
% Important note:
% The bibliography/references need to be compiled with bibtex.
%
%%%%%%%%%%%%%%%%%%%%%%%%%%%%%%%%%%%%%%%%%

%----------------------------------------------------------------------------------------
%	PACKAGES AND OTHER DOCUMENT CONFIGURATIONS
%----------------------------------------------------------------------------------------

\documentclass{beamer}

\usetheme{focus} % Use the Focus theme supplied with the template
% Add option [numbering=none] to disable the footer progress bar
% Add option [numbering=fullbar] to show the footer progress bar as always full with a slide count

% Uncomment to enable the ice-blue theme
%\definecolor{main}{RGB}{92, 138, 168}
%\definecolor{background}{RGB}{240, 247, 255}

%------------------------------------------------

\usepackage{booktabs}% Required for better table rules
\usepackage[style=authoryear, backend=biber, hyperref,sorting=none]{biblatex}
\addbibresource{references.bib}

%----------------------------------------------------------------------------------------
%	 TITLE SLIDE
%----------------------------------------------------------------------------------------

\title{Version Control \& Git}

\subtitle{Teach Me X}

\author{Ewan Gilligan}

\titlegraphic{\includegraphics[scale=0.15]{Images/Git-Logo-Black.png}}



\date{\today}

%------------------------------------------------

\begin{document}


%------------------------------------------------

\begin{frame}
	\maketitle % Automatically created using the information in the commands above
\end{frame}

\begin{frame}{Motivation}
What is Version Control?
    \begin{itemize}
        \item ``Version control is a system that records changes to a file or set of files over time so that you can recall specific versions later'' \parencite{10.5555/2695634}
        % A very naive way of doing this is to have myfile1.txt, myfile2.txt, myfileFINAL.txt. Very cumbersome, many files, and prone to error.
    \end{itemize}
And why is it important?
    \begin{itemize}
        \item Revert changes that mess up your codebase
        \item Humans are bad at managing versions
        \item Separate changes that different users make
        \item Find out who made which changes
        \item ...
    \end{itemize}
\end{frame}

\begin{frame}{Git}
What it is?
\begin{itemize}
    \item Distributed Version Control System (More on this in a second)
    % Previously BitKeeper was used, but they were a commercial company and free of charge use was revoked.
    \item Developed for use with the Linux Kernel in 2005
\end{itemize}
Why you should use it?
\begin{itemize}
    \item Most commonly used Version Control System (VCS) in the world (according to the Stack Overflow Dev Survey)
    \item Commonly used in both open source and commercial software development.
\end{itemize}
Git is centred around the idea of a repository, which contains all the history of a project.
% Some VCS work around changes and store information as a series of change (delta based version control). Git stores it's data as a series of "snapshots" of a filesystem. The entire state of the project is captured in a snapshot (it doesn't store files that haven't changed). Stream of snapshots. This is important later with branching
    
\end{frame}

\begin{frame}{Aside: Local, Distributed, and Centralised VCS}
    Local
    \begin{itemize}
        \item Keeping track of changes to files locally only.
    \end{itemize}
    % Simple, no extra setup required. 
    Centralised
    \begin{itemize}
        \item Single server that contains all versions files
        \item Clients check out files from the central server
    \end{itemize}
    % Much easier to administrate
    % Single point of failure, same issue for local
    Distributed
    \begin{itemize}
        \item Client mirrors entire repository, which includes the history.
    \end{itemize}
    % Each clone of the system is essentially a backup.
    % Git can be used entirely locally, or as distributed.
\end{frame}

\begin{frame}{How to ``git'' help}
There are a few main ways to get help with git.
\begin{itemize}
    \item To get detailed help check the \texttt{man} page
    %either man git <verb>, git help <verb>,  git <verb> -h
    \item For a quick overview add the \texttt{-h} option to the command, e.g. \texttt{git add -h}
    \item Stack Overflow
    % Almost any issue you have with git is likely to have been encountered before.
\end{itemize}
\end{frame}

\begin{frame}{Repository Initialisation}
There are two main ways to create a git repository:
\begin{itemize}
    \item call \texttt{git init} in a local directory
    \item \texttt{git clone} an existing remote repository..
\end{itemize}
This will create a \texttt{.git} directory that stores all the information about the repository. 
% It is worth noting that no files are tracked yet. The command being called clone illustrates the distributed version control at work. You don't just get a working copy, you get a full copy of the the git repo with full history.
\end{frame}

\begin{frame}{Basic Git Workflow}
    \begin{enumerate}
        \item Modify files in the working tree
        % These files are in the modified state now
        \item Stage changes which you want to commit (using \texttt{git add})
        % The files are now staged
        \item Commit the changes you have made (using \texttt{git commit}).
        % The snapshot of the current directory will be saved to the git directory
    \end{enumerate}
    We'll be adding more stages as we go along. 
    \begin{itemize}
        \item You can check what files have been modified with \texttt{git status}.
        \item or using \texttt{git diff}, more on this later.
        % add -s or --short to get a less verbose output.
        \item Files aren't tracked until you \texttt{git add} them.
        % It won't be in the staged state if the file is not tracked. Also note a file is staged exactly how it was when the file itself is staged, so if you add it and then make further changes, these won't be reflected in the commit.
    \end{itemize}
\end{frame}

\begin{frame}{Git-ting rid of unwanted files}
    Not really unwanted, but ones you don't want to track.
    \begin{itemize}
        \item You don't generally want to track things such as large data files, config files, compiled binaries, etc.
        \item Can add these to \texttt{.gitignore} file to stop tracking these.
        \item Standard glob patterns apply to these, but there are a few extra things.
        %Github has quite a few templates for different languages if you want to use these. You can also place additional gitignore files in subdirectories if you want to change what is tracked for subdirectories.
    \end{itemize}
\end{frame}

\begin{frame}{Committing Changes}
    Once you have staged all your changes, it is time to commit them
    \begin{itemize}
        \item Using \texttt{git commit}
        \item Add \texttt{-m "..."} to add a commit message
        \item Excluding the message will open up an editor for you to use to add a commit message
        \item Make sure to include a meaningful message!
        % especially important when working with others, but a good habit to get into.
        \item Using the \texttt{-a} option will stage all files that have changed, so you can skip the \texttt{git add} stage.
        \item You can view the commit history using the \texttt{git log} command
        % There are lots of options, so it is worth looking at the manual for the command.
        % -p --patch
        % --pretty=oneline
        % Can specify your own with format for pretty.
        % --graph not interesting right now, but more so once branches are introduced
        % --since=2.days
        % -S shows commits that changed number of occurrences of a string
        % --grep to grep the commit messages
    \end{itemize}
\end{frame}

\begin{frame}{When things go wrong...}
    You need to be able to undo things when you make a mistake
    %very likely when starting off
    \begin{itemize}
        \item \texttt{git commit --amend} to redo a commit with additional changes
        \item %Really creates a new commit in its place, replacing the old one. Helps avoid commits like fixing typo or forgot this file.
        \item \texttt{git reset HEAD <file>} to unstage a file.
        \item \texttt{git checkout -- <file>} to revert a file to what it was previously
        %both shown by git status
        %There are more commands, these are just the basics.
    \end{itemize}
\end{frame}

\begin{frame}{Remotes}
    So far everything we have done has only made changes to the local copy of the repository. To share our changes with others, we need a remote.
    \begin{itemize}
        \item Remote will be automatically added if you clone from a server, can see which remotes you have with \texttt{git remote}
        %-v for URLS. Can see different for fetch and push, so you can specify different locations.
        \item Origin in the default name given to the server you cloned the repo from.
        % The name origin isn't special, it's just the default.
        \item You can add new remotes with \texttt{git remote add <identifier> <url>}
        \item To get changes from the remote use \text{git fetch <remote>}, usually \texttt{git fetch origin}
        \item To add changes to the remote use \texttt{git push <remote> <branch>} (more on branches next)
        % This will make your changes visible to others.
        \item \texttt{git remote show <remote>} will show details of the remote.
    \end{itemize}
\end{frame}

\begin{frame}{Branching}
    So far we've just been working with a linear series of commits, which isn't ideal for collaborating with others. The solution is branches:
    \begin{itemize}
        \item The snapshots that git stores with a commit contain pointers to the previous commit(s) that directly before this commit.
        % I say commits as if you merge multiple commits together then it'll have multiple parents.
        \item A branch is simply a movable pointer to a commit.
        \item Default branch name is master, but this can be renamed.
        \item Allows you to separate out changes from the master branch, then merge these in when you are done.
    \end{itemize}
\end{frame}

\begin{frame}{Creating and switching branches}
    \begin{itemize}
        \item \texttt{git branch} will list all the current branches.
        \item \texttt{git branch <branch name>} will create a new branch at the current head location.
        % This will point to the current commit. HEAD is just where you currently are
        \item \texttt{git checkout <branch name>} allows you to switch to another branch.
        \item \texttt{git branch -d <branch name>} is used to delete branches.
        % Branches are cheap to create and destroy, as they are just pointers to commits.
    \end{itemize}
\end{frame}

\begin{frame}{Merging Changes}
It would be pretty useless if we couldn't then bring the changes we have made together.
\begin{itemize}
    \item That is what \texttt{git merge <branch name>} is for
    \item Note it merges the branch you specify into the branch you are currently on.
    % So if you want to merge changes into master, then make sure you have master checked out
    \item Git will attempt to merge the two commit histories together into one.
    \item Doesn't always work, so merge conflicts can occur. 
    \item Remember to commit after you've fixed the merge conflicts.
\end{itemize}
\end{frame}

\begin{frame}{Remote Branches}
    The real power of branches comes into play when you have remote branches.
    \begin{itemize}
        \item They have names of the form <remote>/<branch name>, e.g origin/master.
        \item Pushing a branch to the remote will automatically create the remote branch for you.
        \item \texttt{git push <remote> <branch name>} to do this.
        % If they have different names, you can split the names with a colon, the first being your branch name and the remote branch name
        \item \texttt{git checkout <remote branch name>} will automatically add the remote branch as a local branch if it doesn't exist.
        % This is really a shorthand for git checkout --track <remote>/<branch name>, which in turn is a shortcut for git checkout -b branchname remote/branchname
        \item git will then know where to push and pull from, as this branch ``tracks'' the remote.
        \item To get changes from a remote branch, you'll want to fetch and then merge the changes. There is also \texttt{git pull} which is shorthand for that, but this can go wrong.
        % If in doubt, just fetch and merge yourself.
    \end{itemize}
\end{frame}

\begin{frame}{Rebase}
    Merge isn't the only way to combine multiple branches together. There is also \texttt{git rebase <branch to rebase onto>}
    \begin{itemize}
        \item Takes all the changes committed one one branch, and apply these on top of a different branch.
        \item \texttt{rebase } goes back to a common ancestor, takes the diff introduced by each commit, and then applies each change to the branch.
        \item Which of \texttt{rebase} or \texttt{merge} to use really depends on the situation, but good advise is to never rebase anything that you have pushed to somewhere else.
        % People may have based work on those commits, which you have no rebased
        \item Rebase can be good for cleaning up local work before pushing.
    \end{itemize}
\end{frame}

\begin{frame}{Updated Workflow}
    \begin{enumerate}
        \item If you're starting on a new ``bit'' of code, make a new branch.
        \item Modify files in the working tree
        \item Stage changes which you want to commit (using \texttt{git add})
        \item Commit the changes you have made (using \texttt{git commit}).
        \item Push changes to the remote
        \item If you're done on the ``bit'' of code, merge this into the original branch (commonly master).
        % Often you'll want to do what is known as a merge/pull request, which may require permission for you to merge into master after the code has been reviewed and approved.
    \end{enumerate}
\end{frame}

\begin{frame}{Misc Commands}
Here are some commands that didn't really fit in:
\begin{itemize}
    \item \texttt{git rm} will stage a file for removal
    \item \texttt{git mv} to move files. 
    % Git doesn't explicitely track file movement, so it won't know while staging that you moved a file. It'll figure it out after the fact. Like the usual unix mv, you can also use this to rename a file.
    \item \texttt{git tag} to tag important points in the repository history, commonly different release or version points.
    \item Creating alias', which can allow you to rename a command
    % git config --global alias.unstage `reset HEAD --'
    \item You can delete remote branches with \texttt{git push <remote> --delete <branch name>}
\end{itemize}
\end{frame}

\begin{frame}{For the Lazy}
    If you aren't a big fan of the command line, then there are various visual tools for git operations such as:
    \begin{itemize}
        \item git gui
        \item GitKraken (free as part of GitHub student pack)
        \item Github Desktop
        \item ...
    \end{itemize}
    It is still important to know what goes on under the hood, as you might experience a problem these can't solve.
\end{frame}

\begin{frame}{References}
    \printbibliography[heading=none]
\end{frame}




\end{document}
